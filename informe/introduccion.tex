\clearpage
\setcounter{page}{1}

\section{Introducción}

En tiempos de crisis sanitaria como los que atraviesa el país y el mundo en este momento, es necesario poder tomar decisiones que permitan resguardar la salud de la población y mantener el nivel de productividad económica que mitigue los efectos sobre los niveles de desocupación y pobreza. El contexto de \textit{distanciaciamiento social} plantea el problema de \textit{negocio por medio} (NPM) que busca determinar qué locales conviene abrir, dentro de un conjunto de locales de zonas comerciales, en función del \textit{beneficio} económico que aportan, asignado por el gabinete económico, y un valor de \textit{contagio}, asignado por un grupo de expertos de la salud.

Formalmente, dado una secuencia de $n \ge 0$ locales comerciales en orden $L = [ 1, \dots, n]$, el beneficio y contagio $b_i, c_i \in \mathbb{N}_{\ge0}$ de cada local $i \in L$ y el límite de contagio $M \in \mathbb{N}_{\ge 0}$, el problema de NPM consiste en determinar cuál es el máximo beneficio comercial obtenible sin sobrepasar el límite de contagio establecido por los especialistas. Para simplificar, se considera \textit{solución} a todo subconjunto de locales $L' \subseteq L$ y se dice que es \textit{factible} si la suma de los valores de contagio de los locales no supera el límite, es decir, $(i)$ $\sum_{i\in L'}b_i \le M$, y si no cuenta con dos locales colindantes, es decir, si $(ii)$ $(\forall i \in [2,\dots, n]~\wedge~i \in L') : (i-1 \not \in L')$. Además se espera que siempre sea posible al menos poder abrir un local, es decir, $\exists\ i \in [1,\dots,n] : c_i \le M$.

A continuación se exhiben algunos ejemplos con sus correspondientes respuestas esperadas. 

\begin{itemize}
\item $n=4,\ M=40,\ b=[10, 20, 30, 40],\ c=[10,10,10,10]$.

Las soluciones factibles son $L'_1=\{1\},\ L'_2=\{2\},\ L'_3=\{3\},\ L'_4=\{4\},\ L'_5=\{1,3\},\ L'_6=\{1,4\},\ L'_7=\{2,4\}$ y la solución óptima es la 7 con beneficio máximo de 60.
\item Por otro lado, si se tiene $M=20,\ b=[10, 15, 30, 15],\ c=[15,25,10,5]$.

Las soluciones factibles son $L'_1=\{1\},\ L'_2=\{3\},\ L'_3=\{4\},\ L'_4=\{1,4\}$ y la solución óptima es la 2 con beneficio máximo de 30.

\end{itemize}

El objetivo de este trabajo es implementar una solución para NPM utilizando tres técnicas algoritítmicas distintas y evaluar la efectividad de cada una para distintos conjuntos de instancias. En primer lugar se utiliza \textit{fuerza bruta} (FB) que consiste en enumerar todas las soluciones posibles, de manera recursiva, y luego buscar entre las soluciones factibles aquella que sea óptima. Para el algoritmo de \textit{backtracking} se introducen podas para reducir el número de nodos del árbol recursivo. Finalmente, se utiliza almacenamiento en memoria para evitar reprocesar resultados de subproblemas ya calculados. Este proceso se conoce como \textit{memoización} y el algoritmo resultante es el de \textit{programación dinámica} (PD).

En la sección de metodología se introducen y explican los algoritmos de cada una de las técnicas utilizadas en el trabajo junto con las respectivas demostraciones de correctitud y complejidad. Luego, se exponen los experimentos realizados con sus resultados y la respectiva discusión. Por último, se detallan las conclusiones finales del trabajo.
